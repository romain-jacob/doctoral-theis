% !TEX root = ../00_thesis.tex

%-------------------------------------------------------------------------------
\section{Implementation and Scalability of \triscale}
\label{sec:triscale_implementation}
%-------------------------------------------------------------------------------

%-------------------------------------------------------------------------------
\subsection{Python Package}

We implement \triscale in Python ($\approx$1000 lines of code) and make it open source~(\cref{append:triscale_artifacts}).
\triscale's API contains one function for each timescale of the data analysis; \ie the computation of metrics, KPIs, and variability scores.
Docstrings contain detailed information about the functions usage.
Our implementation relies on standard scientific packages such as NumPy~\cite{numpy}, Pandas~\cite{pandas}, SciPy~\cite{scipy}.
As the use of non-parametric statistics is not (yet) widespread, we had to implement some of the statistics used by \triscale (in particular the computation of CI using Thompson's method).
We hope to see these functions integrated in a future release of SciPy.

It is important to produce useful visualizations to support the experimenter.
Thus, we paid a particular attention to the plotting functions in \triscale.
\triscale uses Plotly~\cite{plotly} to create interactive plots: one can zoom in and out in the plots, toggle the visibility of individual traces, read data point values on hover, \etc
All the plots in this chapter are produced using \triscale and are ``clickable'': figures are hyperlinks leading to dynamic versions of the plots.


%-------------------------------------------------------------------------------
\subsection{Scalability of \triscale Data Analysis}
\label{subsec:scalability}

We evaluate the scalability of \triscale with respect to computation time; \ie how does the data analysis time scales with increasing input sizes.
We only consider the time required for performing computations; other outputs such as logs and plots (\eg \Cref{fig:analysis_metric}) are excluded.
The complete scalability evaluation (including data, plots, and discussions) is available as complementary materials~(\cref{append:triscale_artifacts}).
Generally, the computation time for the data analysis in \triscale scales linearly with the input size~(\Cref{table:scalability}): it is fast (less than 1\s for one million data points on a commodity laptop) and overall negligible compared to the data collection time.


\begin{table}
    \centering
    \caption{Scalability evaluation.
    \capt{\triscale data analysis is fast and scales well with increasing input sizes. The most time-consuming element is the convergence test~(\cref{subsec:test_convergence}) which is performed before the computation of metrics. Still, it generally takes less than one second for inputs (\ie the number of raw measurements in a run) of up to one million data points.}}
    \input{\PathTab/triscale_scalability.csv}
    \label{table:scalability}
\end{table}
