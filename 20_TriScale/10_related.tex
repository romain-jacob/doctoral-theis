% !TEX root = ../00_thesis.tex

\pagebreak
%-------------------------------------------------------------------------------
\section{Related Work}
\label{sec:relWorks}
%-------------------------------------------------------------------------------

%-------------------------------------------------------------------------------
% The reproducibility crisis
The reproducibility of experiments and comparability of results are cornerstones of the scientific method.
In recent years, several studies have highlighted the inability of scientists from various disciplines to reproduce their own experimental results~\cite{baker16reproducibility, peng15crisis}, often due to sloppy research protocols and faulty statistical analysis~\cite{boisvert2016Incentivizing, blackburn2016Truth, schmid2014measuring}.
This problem has also been recognized within computer science~\cite{collberg15reproducibility, vitek11systems}, where experiments are seldom reproducible and artifacts rarely shared.


%-------------------------------------------------------------------------------
% Promoting reproducibility
\fakepar{Promoting reproducibility}
To address this ``\mbox{reproducibility} crisis''~\cite{baker16reproducibility}, several efforts aiming to incentivize a rigorous experimentation have gained momentum in computer science, including \eg ACM's badging system \mbox{for publications}~\cite{acmBadges}.
Especially in the networking community -- challenged by the need to carry out experiments on dynamic and uncontrollable conditions~\cite{burchfield09rfjungle, matos18reproducible} -- several workshops~\cite{bajpai18dagstuhl_report, cpsbench18cpsweek, reproducibility17sigcomm}, surveys~\cite{flittner2018artifacts_survey}, guidelines~\cite{bajpai19dagstuhl_guide, saucez2018Thoughts, kritsis2018Tutorial, sevenwaystofail}, as well as teaching activities~\cite{yan17learning} have raised awareness on the reproducibility problem and promoted better experimentation practices.
This large body of work mostly offers \emph{qualitative} statements on how an experiment should be performed and documented.
Such qualitative statements emphasize for example the need to carefully choose when and how often to sample data~\cite{bajpai19dagstuhl_guide}, or suggest which methodology to adopt during performance evaluations~\cite{kritsis2018Tutorial}.
However, there is no guarantee that following these recommendations leads to reproducible results, nor is there a concrete way to assess whether an experiment can be considered reproducible.

None of the existing works provide scientists with \emph{quantitative} answers about how to concretely perform an experiment, \eg how many runs should be completed and how long should they be.
\triscale fills this gap by providing quantitative answers to these questions with an experimental methodology grounded on robust non-parametric statistics.
\triscale also allows to assess and compare the reproducibility of experimental results by computing unambiguous performance indicators and variability scores.

%-------------------------------------------------------------------------------
% Tools supporting repeatability while experimenting
\squarepar{%
  \fakepar{Supporting reproducibility}
  A large number of experimental facilities and tools have been developed in recent years to aid scientists and practitioners in carrying out reproducible networking studies~\cite{nussbaum17testbeds}.
  Testbeds such as EmuLab~\cite{white02emulab} and FlexLab~\cite{ricci07flexlab}, as well as emulation tools such as MiniNet~\cite{handigol12container} and MahiMahi~\cite{netravali2015mahimahi}, enable the creation of artificial network conditions using a given specification or passively-observed traffic.
  Emulated conditions offer a more controlled environment than experiments faced with real-world traffic (\eg by transmitting data over the Internet~\cite{chun03planetlab, berman2014GENI}, cloud~\cite{duplyakin19cloudlab, bolze06grid5000}, or wireless interfaces~\cite{adjih15iotlab, ganu05orbit, massouri14cortexlab}).
  Still, they suffer from performance variability caused by the underlying hardware and software components, which hampers reproducibility~\cite{maricq2018Taming}.
  To overcome these problems, several solutions have been proposed~\cite{edwards15testbeds}: \eg revisiting  operating system libraries~\cite{tazaki13directcode}, using virtualization~\cite{handigol12container, kannan18bnv, koponen14nvp}, adaptable profiles~\cite{ricci2015Apt}, and fault patterns~\cite{angainor_website}.
  Other tools have been developed to support mobility experiments~\cite{cho16phantomnet_repeatability, banerjee2015PhantomNet}, maximize the repeatability of interference generation~\cite{schuss19jamlabng}, and enable researchers to consistently evaluate congestion control schemes or transport protocols~\cite{yan18pantheon}.
  Other works model the execution of experiments, and uses such models to quantify the similarity between different runs~\cite{sharma17framework,ferreira2017MetaAnalysis}.

  While all aforementioned tools aim to improve reproducibility \emph{during} the experiments, \triscale assists researchers \emph{before} and \emph{after} their execution.
  It does so by informing about the number and length of runs necessary to obtain a sufficient statistical significance, as well as by computing a score quantifying the variability of the results.
  Hence, \triscale complements the existing body of literature promoting and enhancing reproducibility in networking research.%
}
