% !TEX root = ../00_thesis.tex

% ------------------------------------------------------------------------------
\section{Implementing \TTW}
\label{sec:ttw_implementation}
% ------------------------------------------------------------------------------

\TTW is a wireless \CPS design composed of two main building block:\linebreak
% (i)~
% \inlineitem
(i)~the system's communication backbone, which we call \TTnet ~(\cref{sec:ttw_overview}), and
% \linebreak
% (ii)~
% \inlineitem
(ii)~the \TTW scheduler, which synthesizes the timing of execution of all tasks, messages, and communication rounds~(\cref{sec:single_mode,sec:multi_mode}).
In this section, we describe how we implement \TTnet and the \TTW scheduler to obtain a fully functional \TTW system.
The performance of the implementation is discussed in \cref{sec:ttw_evaluation_sched,sec:ttw_evaluation_implem}.

% ------------------------------------------------------------------------------
\subsection{Implementation of \TTnet}
\label{subsec:implem_ttnet}

We implement \TTnet on embedded platforms built using the Dual-Processor Platform (\DPP) concept.
The \DPP links two arbitrary processors with a processor interconnect called \bolt~\cite{sutton2015Bolt}, which provides predictable asynchronous message passing between the two processors using message queues with first-in-first-out (FIFO) semantics, one for each direction.%
%
\footnote{Refer to the Introduction chapter for more details about the \DPP concept.}
%
The \DPP architecture is compatible with \TTW's system model, which assumes that a node is capable of performing one task execution and one message transmission simultaneously~(\cref{sec:model}):
We dedicate one processor (a TI MSP432~\cite{msp432}) to the execution of tasks and another one (a TI CC430 SoC~\cite{CC430F6137}) to wireless communication.
The \DPP platform we used is illustrated in~\cref{fig:dpp2-cc430}.

To implement the \TTnet network stack, we leverage the \baloo design framework~(introduced in \cref{ch:baloo}).
In particular, we use the ``static configuration'' mode of \baloo~(\cref{sec:adv_features}):
\TTW produces scheduling tables offline, which can be loaded in the nodes' memory to limit the runtime communication overhead~(\feature{Efficiency}).
The \TTnet beacons~(\cref{sec:ttw_overview}) are sent as \baloo control packets, using the customizable ``user-bytes'' field.%
%
\footnote{\href{https://github.com/ETHZ-TEC/Baloo/wiki/Baloo-control-packet}{github.com/ETHZ-TEC/Baloo/wiki/Baloo-control-packet}}


\begin{table}
  \centering
  \caption{Worst-case execution time of \bolt \opread and \opwrite functions}
  \label{table:boltAPIwcet}
  {\smaller \input{\TablePath/boltWCET.csv}}
\end{table}


To synthesize the schedules, one must know how long a communication round lasts.
For this purpose, the predictability of the \baloo framework is very beneficial: based on the implementation parameters, one can derive a precise estimate of the maximum possible length of a round.%
%
\footnote{The \TTnet model and its evaluation are presented in \cref{sec:ttw_evaluation_implem}.\label{footnote:ttnet}}
%
This model must account for the time to read and write messages over \bolt, the \DPP processor interconnect.
The \bolt API functions have formally verified semantics and bounded execution times~\cite{sutton2015Bolt}.
In our implementation, the two processors read and write over \bolt using the maximally supported SPI frequency of 4\MHz, leading to the worst-case execution times shown in \cref{table:boltAPIwcet}.
On the application side, we include the time to read and write messages within the WCET of tasks; in other words, we consider that a task ``starts'' when the processor initiates the \bolt \opread function, and terminates once the \bolt \opwrite function is completed.%
%
\footnote{Assuming that the task has some input and output messages.}
%
On the communication side, incoming messages are read from \bolt before the rounds. This is performed during the so-called ``pre-process'', which is scheduled before each \baloo round.%
%
\footnote{\href{https://github.com/ETHZ-TEC/Baloo/wiki/Baloo-pre-post-processes}{github.com/ETHZ-TEC/Baloo/wiki/Baloo-pre-post-processes}}
%
The messages received from the wireless network are written over \bolt directly after the communication slot, in \baloo's \texttt{on\_slot\_post()} callback function~(\cref{sec:baloo_implementation}).
Our implementation of \TTnet is available in the \baloo repository~\cite{repoBaloo}.



% ------------------------------------------------------------------------------
\subsection{Implementation of the \TTW Scheduler}
\label{subsec:implem_sched}

We implement the \TTW scheduler using Matlab~\cite{Matlab} and we use the Gurobi solver~\cite{Gurobi} for synthesizing the schedules.
All scripts are publicly available,%
%
\footnote{Although Matlab and Gurobi are commercial software, free academic and/or student licenses are currently available from the software vendors.}
%
together with details of the MILP formulation~(\cref{appendix:ttw_artifacts}).
In this section, we describe how to run the \TTW scheduler.

% \begin{itemize}
%
%   \item
  There are four high-level communication parameters that the user must specify in the \texttt{multimode\_main.m} file:
  \begin{itemize}[nosep]
    \item \customBox{$L$} The message payload size (in\bytes)
    \item \customBox{\nslotsmax} The maximal number of slots per round
    \item \customBox{$N$} The number of message transmissions in a Glossy flood~\cite{ferrari2011Glossy}
    \item \customBox{$H$} The estimated network diameter (in number of hops)
  \end{itemize}
  %
  % \item
  The \texttt{loadRoundModel.m} file contains the parameters of the \TTnet implementation~(\cref{subsec:implem_ttnet}). Combined with the high-level parameters described above, this allows the scheduler to compute the length of communication rounds.$^{\text{\ref{footnote:ttnet}}}$
  %
  % \item
  Finally, the user must specify the system's operation modes, applications, tasks, and messages, which is done by filling cell arrays in the \texttt{system\_configuration.m} and \texttt{mode\_configuration.m} files.
%
% \end{itemize}

\pagebreak

The scheduler supports user-defined constraints on the task and message offsets. These constraints must have the following form
\begin{equation}
  \sum_k \; \alpha_k*\mathit{id}_k.o \quad\mathit{`sign'}\quad \beta
\end{equation}
where $\mathit{id}$ is the unique identifier of a task or a message, $\alpha$ is a constant multiplier, $\mathit{sign}$ can be specified as `$=$' or `$<$', and $\beta$ is the right-hand side term of the constraint. The constraint may contain an arbitrary number of left-hand terms, denoted here by $k$.
These constraints are specified as tuples of the form
$  \{\,\{\mathit{id}_k,\, \alpha_k \}_k\,,\; \mathit{sign}\,,\; \beta \,\}$, as shown in~\cref{fig:user_constraints}; in that example, the user-defined constraints force some tasks to have the same offset; in other words, these tasks will execute simultaneously.
The user-defined constraints are automatically added to the MILP formulation; if the problem is feasible, the schedule is guaranteed to satisfy them.
Once all inputs have been specified, the schedules for all operation modes are synthesized and displayed by running the \texttt{multimode\_main.m} script.

\begin{figure}
{\smaller
\begin{lstlisting}%[caption = {Example of specification of user-defined constraint}]

CustomConstaints = { ...
    { {{'T_loc_stab1', 1},{'T_loc_stab2', -1}}, '=', 0}, ...
    { {{'T_loc_stab1', 1},{'T_loc_stab3', -1}}, '=', 0}, ...
    { {{'T_loc_stab1', 1},{'T_loc_stab4', -1}}, '=', 0}, ...
    { {{'T_loc_stab1', 1},{'T_loc_stab5', -1}}, '=', 0}, ...
    };
\end{lstlisting}}
\caption{Example of specification of user-defined constraints.
\capt{The generic formulation let the user define any linear constraint between the offsets of tasks and messages in the system. In this example, the constraints force all tasks \textup{\texttt{T\_loc\_stabX}} to have the same offset; in other words, these tasks will execute simultaneously.}}
\label{fig:user_constraints}
\end{figure}
