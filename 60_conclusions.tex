% !TEX root = 00_thesis.tex

\chapter{Conclusions and Outlook}
\label{ch:conclusions}

Cyber-Physical Systems (\CPS) are believed to be the vector of the next computing revolution~\cite{rajkumar2010CPS}. In recent years, the buzz words have changed: it is now all about the ``Internet of Things'' or ``Industry 4.0'' -- different names, same concept.
We are envisioning a world where devices with computing and wireless communication capabilities are deeply embedded into our lives, in our environments, and even within ourselves.
However, the current state of technology is not quite yet able to realize these grand visions.
In this dissertation, we considered certain challenges related to communication in ad-hoc networks; or as we say today: ``on the edge''.
One important challenge is to support energy efficient and reliable communication within a network of mobile nodes, such as teams of robots or drones.
It was recently shown that the technique referred to as \emph{synchronous transmissions} (\ST) has a lot of potential to tackle this challenge.
Indeed, \ST can be leveraged to perform network-wide flooding in a stateless fashion; that is, packets are sent to all nodes in the network, in a way that does not depend on the current topology. And since the topology does not matter, supporting mobility comes essentially for free.


Despite being a potential enabler for a whole new class of mobile \CPS applications, \ST has not yet been used much outside academia. We identify at least three reasons for that:
\begin{itemize}

  % (i)~
  \item
  \ST requires a very precise control of the hardware; in particular, the timing of radio operations is critical~(\cref{sec:ST}).
  This makes \ST complex to integrate into larger systems. As there is a lack of tools and methods to facilitate this integration, using \ST requires an expert knowledge that the typical system designer does not have.
  \item
  % (ii)~
  It is easier to believe what we can see, and the efficiency and reliability of \ST often come across as too-good-to-be-true. While the academic community is now generally convinced, industry demands some proof-of-concepts showing that \ST does fulfill its promises, when put to the test.
  \item
  % (iii)~
  Finally, it is paramount to confidently evaluate and estimate expected performance.
  The challenge of reproduciblility in networking experiments is often debated, and there seems to be a general lack of trust in the results presented in academic papers.
  It is not saying that researchers lie about their experiment results, but would the system perform as well in another scenario? Or in another network? Or with other parameters?
  The inability of answering these questions  is problematic.

\end{itemize}

\section{Contributions}
This dissertation makes three main contributions to foster the adoption of \ST.

\begin{description}

    \item[\baloo]
    We designed \baloo, a framework facilitating the implementation of network stacks based on \ST.
    It is meant as a tool for system designers to test and experiment with \ST without having to be versed into the details of radio management: users implement their protocol through a simple yet flexible API while \baloo handles all the complex low-level operations based on the users' inputs.

    \item[\DRP and \TTW]
    We proposed two different system designs -- the Distributed Real-time Protocol (\DRP) and Time-Triggered Wireless (\TTW) -- which demonstrate for the first time that end-to-end real-time guarantees can be obtained in \CPS using low-power wireless technology by leveraging \ST.
    Our systems explore different parts of the design space:
    \DRP uses contracts to maximize the flexibility of execution between distributed tasks, whereas
    \TTW statically co-schedules all task executions and message transfers to minimize end-to-end latency.
    In particular, together with project partners, we used the \TTW design to implement the first demonstrator of feedback-control of fast physical systems (multiple inverted pendulums) across a multi-hop network using low-power wireless technology~\cite{mager2019Demo}.

    \item[\triscale]
    With the design of \triscale, we worked towards more rigorous and reproducible experimental networking research.
    For the first time, we went beyond simple guidelines and proposed a concrete methodology for designing networking experiments and analyzing their data, which allowed us to proposed the first formalized definition of reproducibility in the context of networking experiments.
    The driving principle of \triscale is to rationalize the evaluation process: \ie justify what is required to do and what data to collect in order to support one's performance claims.


\end{description}

% \pagebreak
\section{Future Developments}

\fakepar{Benchmarking Wireless Protocols}
Our work on \triscale stemmed from discussions in the low-power wireless community regarding the need for a benchmark to compare networking protocols~\cite{boano2018IoTBench}.
As we were reflecting on how to design such a benchmark, the need for a more rigorous experimental methodology became obvious; a need that \triscale tries to fill.
We can now return to our initial objectives and attempt to realize the vision of IoTBench: a benchmark to thoroughly and confidently compare the performance of wireless networking protocols~\cite{IoTBench_website}.

\fakepar{Going further with \ST}
With the design of \baloo, we attempted to make \ST more accessible; an attempt that appears to be successful. Less than a year after the initial paper, the first independent studies using \baloo have been published~\cite{spina2019XPC}.
There are many opportunities for future developments of the framework; the most natural being the port of \baloo to other platforms.
It has been shown that the principle of \ST also work on other physical layers that IEEE802.15.4 (\eg Bluetooth~\cite{alnahas2019BlueFlood} and LoRa~\cite{wegmann2018Reliable}).
To investigate this further, a port to the LoRa-compatible SemTech SX1262 chip~\cite{semtechSX1262} is currently under development.
Another port to the Bluetooth-compatible nRF52840 Dongle~\cite{nRF52840} is planned in a near future.
%
These would allow to experiment with \ST-based networking on different physical layers and, by leveraging (hopefully upcoming) wireless protocol benchmarks, we would be able to objectively compare the performance trade-offs of these different technologies in a wide range of scenarios and applications.

\fakepar{Dependable networking}
One important limitation of the system designs presented in this dissertation is the reliance on a central authority, which we call \emph{host}, in order to coordinate communication within the network.
This creates a single point of failure: if the host should fail (or be jammed), the entire network would stop its operation.
For any safety-critical applications, this is not acceptable.
It is therefore important to work on system designs that would ``distribute the responsibility'' of the host.
Recent contributions provide consensus primitives in low-power networks~\cite{spina2019XPC, alnahas2019BlueFlood,poirot2019Paxos}, an important piece for fault-tolerance in distributed systems.
However these works still rely on a central authority for elementary network functions, such as time synchronization.
More efforts are required to designed truly dependable wireless networks.

\newpage
\section{Affirming Oneself as a Researcher}

Pursuing a doctorate degree is not only about writing a dissertation: it is a fundamental training step towards affirming oneself as a researcher.
The shaping of my own researcher identify has been strongly influenced by this statement%
\footnote{Erin McKiernan. \href{http://dx.doi.org/10.6084/m9.figshare.954994}{10.6084/m9.figshare.954994}}
\begin{center}
  ``\emph{If I am going to "make it" in science, it has to be on terms I can live with.}''
\end{center}
I cannot agree more and this idea now impacts many aspects of my work.\linebreak
For me, this translates into trying to follow the principles generally referred to as ``Open Science''. I hope this can be perceived in this dissertation.
In practice, this implies for example favoring open and free software over commercial tools; this is why the earlier works from the dissertation used Matlab, while the later ones are based on Python.
Moreover, I try to systematically release data and code with all publications, aiming to make any plot and experiment reproducible by others, which I believe should be a standard in science.

I recently wrote down my own objectives and expectations regarding the way I intend to do research; this has materialized into a ``Pledge to Open Science'' which is publicly available on my personal webpage.%
%
\footnote{\href{http://www.romainjacob.net/pledge-to-open-science/}{www.romainjacob.net/pledge-to-open-science}}
%
I will do my best to live and work by this principles, because I believe this is the right thing to do. \linebreak
We shall see if that will be good enough to ``make it''.
