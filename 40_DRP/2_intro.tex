% !TEX root = ../00_thesis.tex

\section{Problem Setting}
\label{sec:drp_intro}

Cyber-physical systems~(\CPS) tightly integrate components for sensing, actuating, and computing into distributed feedback loops to directly control physical processes~\cite{rajkumar2010CPS}. The successful deployment of \CPS technology is widely recognized as a grand challenge to solving a number of societal problems in domains ranging from healthcare to industrial automation. To reach into new areas and realize systems with unprecedented capabilities, there is a trend toward increasingly smaller and autonomously powered \CPS devices that exchange data through a low-power wireless communication substrate. As many \CPS applications are mission-critical and physical processes evolve as a function of time, the communication among the sensing, actuating, and computing elements is often subject to real-time requirements, for example, to guarantee stability of the feedback loops~\cite{stankovic2005Opportunities}.

These real-time requirements are often specified from an end-to-end application perspective. For example, a control engineer may require that sensor readings taken at time $t$ are available for computing the control law at $t + D$, where the relative deadline~$D$ is derived from
the application requirements; \eg the maximum tolerable delay between a sensing and a control task, where these tasks are typically executed on physically distributed devices.

Meeting end-to-end deadlines is non-trivial because data transfers between application tasks involves multiple other tasks (\eg operating system, networking protocols) and shared resources (\eg memories, system buses, wireless medium).
The entire transmission chain of the data throughout the system must be taken into account to enable end-to-end real-time guarantees.

\begin{research_questions}
  \begin{description}
    \item[Question 1]
    Can we provide end-to-end real-time guarantees between distributed applications in a wireless \CPS?

    \item[Question 2]
    Can we do so while preserving runtime adaptability and flexibility in the timing of task executions?
  \end{description}
\end{research_questions}

\squarepar{%}
  \fakepar{The problem}
  Enabling real-time communication {between network interfaces} of sources and destinations in a low-power wireless network has been studied for more than a decade~\cite{lu2002RAP,stankovic2003Realtime,he2003SPEED}. Today, standards such as WirelessHART~\cite{wirelessHART} and ISA100.11a~\cite{isa100} for control applications in the process industries already exist~\cite{watteyne2017Teaching}, and considerable progress in real-time transmission scheduling and end-to-end delay analysis for WirelessHART networks has been made~\cite{saifullah2010RealTime,saifullah2015EndtoEnd}.

  Unfortunately, wireless real-time protocols such as WirelessHART~\cite{wirelessHART} or \blink~\cite{zimmerling2017Blink} only provide guarantees for message transmissions between \emph{network interfaces}. These protocols do not handle the application schedules; at the source, the message release is typically assumed periodic; at the destination, nothing guarantees that the application will process the message in time.

  Providing end-to-end guarantees between \emph{distributed application} (\textbf{Question~1}) demands to combine a wireless real-time protocol with the rest of the system; \ie consider application schedules and handle interference on shared resources.%
}

\fakepar{The challenge}
To support a broad spectrum of \CPS applications, a solution to this problem should fulfill the following requirements.
\begin{features}

  \item[Timeliness]
  All messages received by the destination application meet their end-to-end deadlines.

  \item[Reliability]
  All messages received at the wireless network interface are {successfully delivered} to their destination application (\ie no buffer overflows).

  \item[Adaptability]
  The system adapts to dynamic changes in traffic requirements at runtime.

  \item[Composability]
  Existing hardware and software components can be freely composed to satisfy specific application's needs, without altering the properties of the integrated parts.

  \item[Efficiency] The solution scales to large system sizes and operates efficiently with regard to limited resources such as energy, wireless bandwidth, computing capacity, and memory.

\end{features}

A major challenge in meeting these requirements is to funnel messages in real-time through tasks that run concurrently and access shared resources.
Interference on such resources can delay tasks and communication arbitrarily, therefore hampering
\feature{Timeliness}, \feature{Reliability}, and \feature{Composability}.

\fakepar{Our solution}
In this chapter, we present a real-time wireless CPS that tackles interference on shared resources by defining (minimal) constraints on the application schedules.
This is achieved by combining a predictable device architecture with a real-time scheduler for the entire system.
\begin{description}
  \item[Predictable device architecture]
  We use the Dual-Processor Platform (\DPP) concept, presented in Introduction (\cref{ch:introduction}). The \DPP dedicates a communication processor (\cp) exclusively to the real-time network protocol and executes all other tasks on an application processor (\ap).
  The \DPP is based on the \bolt interconnect~\cite{sutton2015Bolt}, which decouples two processors in the time, power, and clock domains, while allowing them to asynchronously exchange messages within predictable time bounds.%
  \footnote{Refer to the Introduction~ (\cref{sec:dpp}) for more details.}

  Thus, on each device, we decouple the communication and application tasks, which can be independently invoked in an event- or time-triggered fashion.
  The \DPP concept guarantees the faithfulness of the network interface (\feature{Reliability}), supports \feature{Composability}, and leverages the recent trend toward ultra low-power multi-processor architectures, which can be chosen individually to match the needs of the application and the networking protocol respectively (\feature{Efficiency}).

  \item[Real-time scheduler]
  We design the \DRPLong (\DRP), a scheduler that provably guarantees that all messages received at the application interfaces meet their end-to-end deadlines (\feature{Timeliness}) and that message buffers along the data transfers do not overflow~(\feature{Reliability}).

  To accomplish this while being adaptive to unpredictable changes~(\feature{Adaptability}), \DRP dynamically establishes at runtime a set of {contracts} based on the current traffic demands in the system. A contract determines the mutual obligations in terms of
  % \begin{itemize}
    % \item
    (i)~minimum service provided, and
    % \item
    (ii)~maximum demand generated
  % \end{itemize}
  between the networking protocol and an application.
  \DRP contracts define time bounds that can be analyzed to ensure that end-to-end deadlines are met, while
  preserving flexibility in the timing of distributed task executions (\textbf{Question 2}).

\end{description}

The contributions of this chapter are summarized below.
\begin{itemize}
  \item
  We design \DRP, a wireless \CPS system that provably provides end-to-end real-time guarantees between distributed applications. \DRP does so by harnessing the benefits of synchronous transmissions~(\cref{sec:ST}) and building upon the \blink real-time scheduler~\cite{zimmerling2017Blink} and the Dual-Processor Platform architecture~(\cref{sec:dpp}).

  \item
  We simulate \DRP execution to demonstrate that the provided bounds are both safe and tight.

  \item
  We implement \DRP on embedded hardware and showcase that the protocol works as expected.

  \item
  We make our implementation of \DRP publicly available, which includes the \blink scheduler for LWB~\cite{ferrari2012LWB}.

\end{itemize}
