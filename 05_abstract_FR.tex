% !TEX root = 00_thesis.tex
\chapter{Résumé}

% Context
%% CPS
\startsquarepar
Les systèmes cyber-physiques (\CPS, de l'anglais \emph{Cyber-Physical Systems}) sont des systèmes où des éléments informatiques interagissent avec leur environnement : ils collectent certaines informations sur leur environnement, traitent ces informations et agissent en conséquence, ce qui modifie l'état de l'environnement.
Monter automatiquement le chauffage lorsque la température baisse est l'un des exemples les plus simples de \CPS.
%
La situation devient plus complexe lorsque les applications sont distribuées entre des systèmes embarqués à
faible consommation d'énergie, qui sont supposés fonctionner de façon autonome durant plusieurs années.
Une communication sans-fil fiable et à basse consommation devient alors essentielle.
%
De plus, des délais maximum sur l'exécution de différentes opérations sont souvent imposés. Un système capable de garantir ces délais est appelé un système temps-réel.
%
Des \CPS sans-fil  capablent de fournir des garanties temps-réel tout en utilisant des technologies de communication à basse consommation sont souhaitables mais difficiles à concevoir.
%% ST
Ces dernières années, une technique appelée transmission synchrone (\ST, de l'anglais \emph{synchronous transmissions}) a été utilisée pour communiquer de façon fiable et à basse consommation dans les réseaux multi-sauts.
\linebreak
En un mot, le principe de la \ST est d'autoriser différents appareils à transmettre un paquet durant le même intervalle de temps ; la communication a de bonnes chances de réussir si les transmissions sont suffisamment bien synchronisées.
La \ST peut être utilisée pour réaliser en un temps donné n'importe quel broadcast multi-saut, \cad une communication depuis un appareil à tous les autres ; une propriété très intéressante pour concevoir un système temps-réel.
\stopsquarepar

\startsquarepar
Bien que le potentiel de la \ST soit reconnu par la communauté académique réseaux, cette technique a été pour l'instant peu utilisée pour la conception de \CPS.
Au moins trois problèmes limitent l'adoption de la \ST dans ce domaine.
\linebreak
\smallBox{(i)}La \ST est difficile à utiliser à cause des exigences strictes de synchronisation, de l'ordre de la \us.
Il manque des outils facilitant l'usage de la \ST par des ingénieurs \CPS, qui souvent ne sont pas des experts en communication.
\linebreak
% \\
\smallBox{(i)}Il y a peu d'exemples illustrant l'utilisation de la \ST pour des applications \CPS ; les travaux académiques sur la \ST sont souvent plus focalisés sur les aspects communication que application.
Il manque de preuves de concept convaincantes démontrant l'intérêt de la \ST pour des applications \CPS.
% There are only few examples showcasing the use of \ST for \CPS applications and academic works based on \ST tend to focus on communication rather than applications. Convincing proof-of-concept \CPS applications are missing.
\linebreak
\smallBox{(i)}La variabilité inhérente de l'environnement sans-fil rend difficile l'évaluation des performances. L'absence d'une méthode établie menace la reproductibilité des expériences et limite la confiance dans les performances annoncées.
\stopsquarepar

\pagebreak
\startsquarepar
Ainsi, nous avons développé des outils et méthodes pour faciliter l'évaluation de protocoles sans-fil et l'implémentation de \CPS utilisant la \ST.
De plus, nous avons tiré parti de la \ST pour concevoir deux \CPS visant différentes classes d'applications temps-réel.
Cette dissertation présente ces contributions.
\stopsquarepar

\begin{itemize}

  % =============================== %
  \item
  \startsquarepar
  Dans le chapitre \ref{ch:triscale}, nous proposons de concevoir et d'analyser les expériences d'évaluation de performance pour les protocoles réseaux en utilisant une méthodologie concrète, rationnelle et statistiquement robuste.
  Nous implémentons cette méthodologie dans un framework appelé \triscale, qui permet d'obtenir des performances avec un niveau de confiance quantifiable.
  % \linebreak
  De plus, nous tirons parti de \triscale pour proposer la première définition formelle de reproductibilité appliquée aux expériences pour les protocoles réseaux.
  \stopsquarepar

  % =============================== %
  \item
  % \startsquarepar
  Le chapitre \ref{ch:baloo} présente \baloo, un framework pour la conception de protocoles réseaux basés sur la \ST.
  L'utilisateur implémente son protocole via l'interface fournie par \baloo, qui prend en charge la gestion des opérations complexes telles que garantir la synchronisation nécessaire pour la \ST.
  Nous montrons que \baloo est suffisamment flexible pour implémenter un large panel de protocoles pour un coût minime en termes d'utilisation mémoire et de consommation d'énergie.
  % \stopsquarepar

  % =============================== %
  \item
  Enfin, nous concevons et implémentons deux \CPS sans-fil utilisant la \ST :
  {\setlength{\parskip}{0pt}%
    \begin{itemize}[nosep]
      \item
      le Distributed Real-time Protocol (\DRP) utilise le concept de contrat pour maximiser la flexibilité entre les tâches à exécuter~(Chapitre~\ref{ch:drp}) ;
      \item
      Time-Triggered Wireless (\TTW) planifie statiquement tous les échanges de paquets et les exécutions de tâches de façon à minimiser la latence de bout en bout entre les tâches~(Chapitre~\ref{ch:ttw}).
    \end{itemize}
    \startsquarepar
    Nous démontrons que des garanties temps-réel peuvent être fournies de façon fiable et efficace en énergie.
    De plus, \TTW supporte des latences de l'ordre de dizaines de \ms, ce qui est suffisant pour contrôler en boucle fermée des pendules inversés ; une référence pour les applications de contrôle et de robotique.
    \stopsquarepar
  }

\end{itemize}

% Conclusion
\startsquarepar
Dans cette dissertation, nous montrons que la \ST permet de satisfaire les exigences des \CPS sans-fil temps-réel.
De plus, nous facilitons l'implémentation de tels systèmes avec \baloo, un framework qui rend la \ST accessible pour le non-expert.
Enfin, \triscale est un élément important pour améliorer la confiance dans les performances des protocoles réseaux.
À partir de \triscale, il serait utile de définir, pour différentes classes d'applications, des problèmes de références pour l'évaluation de futurs \CPS.
In fine, il est nécessaire d'évoluer depuis les preuves de concept vers des applications de \CPS sans-fil dans le monde réel. Cela serait facilité par le portage de \baloo sur des systèmes embarqués plus récents et plus puissants, ce qui améliorerait les performances atteignables.
\stopsquarepar
